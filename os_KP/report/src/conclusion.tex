\section{Выводы}

В ходе выполнения курсового проекта была реализована многопользовательская игра "Быки и коровы" с использованием именованных pipe'ов для межпроцессного взаимодействия:

\begin{enumerate}
    \item \textbf{Именованные pipe'ы (FIFO)}: Освоено создание и использование именованных pipe'ов для организации межпроцессного взаимодействия между сервером и клиентами. Сервер создает pipe для приема сообщений, каждый клиент создает свой pipe для получения ответов.
    
    \item \textbf{Мультиплексирование ввода-вывода}: Реализована обработка множественных соединений с использованием системного вызова \texttt{select()}, позволяющего одновременно ожидать данные из нескольких файловых дескрипторов. Это обеспечивает эффективную обработку сообщений от нескольких клиентов.
    
    \item \textbf{Неблокирующий ввод-вывод}: Все pipe'ы открываются в неблокирующем режиме (\texttt{O\_NONBLOCK}), что позволяет использовать \texttt{select()} для проверки готовности данных без блокировки выполнения программы.
    
    \item \textbf{Протокол обмена сообщениями}: Разработан простой текстовый протокол с разделителем \texttt{|} для обмена сообщениями между сервером и клиентами. Протокол поддерживает подключение, отправку предположений, получение результатов и отключение.
    
    \item \textbf{Обработка сигналов}: Реализована корректная обработка сигналов SIGINT и SIGTERM для graceful shutdown сервера и клиентов. Используется функция \texttt{atexit()} для автоматической очистки ресурсов при завершении работы.
    
    \item \textbf{Логика игры}: Реализован алгоритм вычисления быков (правильные буквы на правильных позициях) и коров (правильные буквы на неправильных позициях) с учетом того, что каждая буква может быть использована только один раз.
    
    \item \textbf{Валидация данных}: Реализована проверка корректности входящих сообщений и предположений игроков, включая проверку формата, длины и допустимых символов.
    
    \item \textbf{Обработка отключений}: Реализована корректная обработка отключения игроков во время игры. Сервер помечает отключенных игроков как неактивных, удаляет их pipe'ы и уведомляет остальных игроков.
\end{enumerate}

Программа успешно выполняет все поставленные задачи и демонстрирует корректную работу механизмов межпроцессного взаимодействия через именованные pipe'ы. Реализованная система позволяет эффективно организовать многопользовательскую игру с централизованным управлением на сервере и независимыми клиентскими процессами.
