\section{Описание программы}

Проект состоит из сервера и клиентов, взаимодействующих через именованные pipe'ы (FIFO). Сервер управляет игровой сессией, обрабатывает подключения игроков и вычисляет результаты предположений. Клиенты подключаются к серверу, отправляют предположения и получают результаты.

Сервер создает именованный pipe \texttt{/tmp/bulls\_cows\_server} для приема сообщений от клиентов. Каждый клиент создает свой именованный pipe по шаблону \texttt{/tmp/bulls\_cows\_client\_\%d}, где \%d - ID игрока. Сервер открывает pipe клиента для отправки ответов.

\subsection{Используемые системные вызовы и функции}

\begin{description}
    \item[\texttt{mkfifo(const char *pathname, mode\_t mode)}] Создает именованный pipe (FIFO) для межпроцессного взаимодействия. Используется для создания серверного pipe и клиентских pipe'ов.
    
    \item[\texttt{open(const char *pathname, int flags)}] Открывает именованный pipe для чтения или записи. Сервер открывает pipe в неблокирующем режиме (\texttt{O\_NONBLOCK}) для работы с \texttt{select()}.
    
    \item[\texttt{read(int fd, void *buf, size\_t count)}] Читает данные из pipe. Используется сервером для получения сообщений от клиентов и клиентами для получения ответов от сервера.
    
    \item[\texttt{write(int fd, const void *buf, size\_t count)}] Записывает данные в pipe. Используется для отправки сообщений между процессами.
    
    \item[\texttt{select(int nfds, fd\_set *readfds, fd\_set *writefds, fd\_set *exceptfds, struct timeval *timeout)}] Мультиплексирует ввод-вывод, позволяя одновременно ожидать данные из нескольких файловых дескрипторов. Используется для обработки сообщений от нескольких клиентов и пользовательского ввода.
    
    \item[\texttt{unlink(const char *pathname)}] Удаляет именованный pipe из файловой системы. Вызывается при завершении работы для очистки ресурсов.
    
    \item[\texttt{close(int fd)}] Закрывает файловый дескриптор. Используется для освобождения ресурсов после работы с pipe'ами.
    
    \item[\texttt{signal(int signum, sighandler\_t handler)}] Устанавливает обработчик сигналов. Используется для корректного завершения работы при получении SIGINT или SIGTERM.
    
    \item[\texttt{atexit(void (*function)(void))}] Регистрирует функцию, вызываемую при нормальном завершении программы. Используется для очистки ресурсов.
\end{description}

\subsection{Обработка ошибок}

Программа обрабатывает следующие ошибки:
\begin{itemize}
    \item Ошибки создания именованных pipe'ов (\texttt{mkfifo()} возвращает -1)
    \item Ошибки открытия pipe'ов (\texttt{open()} возвращает -1)
    \item Ошибки чтения/записи (\texttt{read()}/\texttt{write()} возвращают -1)
    \item Разрыв соединения (EOF при чтении из pipe)
    \item Некорректные сообщения от клиентов (неверный формат, отсутствие разделителей)
    \item Некорректные предположения (неверная длина слова, недопустимые символы)
    \item Превышение максимального количества игроков
\end{itemize}

Все ошибки обрабатываются через проверку возвращаемых значений системных вызовов и вывод сообщений об ошибках через \texttt{perror()}, \texttt{fprintf(stderr, ...)} и макросы логирования \texttt{LOG\_ERROR}, \texttt{LOG\_INFO}, \texttt{LOG\_DEBUG}.
