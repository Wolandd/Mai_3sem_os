\section{Метод решения}

\subsection{Общее описание алгоритма}

Проект представляет собой многопользовательскую игру "Быки и коровы" с использованием именованных pipe'ов (FIFO) для межпроцессного взаимодействия между сервером и клиентами.

Сервер выполняет следующие действия:
\begin{enumerate}
    \item Создает именованный pipe для приема сообщений от клиентов
    \item Генерирует случайное слово из словаря
    \item Ожидает подключения игроков через именованные pipe'ы
    \item Обрабатывает сообщения от клиентов: подключение, предположения, отключение
    \item Вычисляет количество быков и коров для каждого предположения
    \item Рассылает результаты всем активным игрокам
    \item Определяет победителя и завершает игру
\end{enumerate}

Клиент выполняет следующие действия:
\begin{enumerate}
    \item Создает именованный pipe для получения сообщений от сервера
    \item Подключается к серверу, отправляя сообщение CONNECT
    \item В цикле читает ввод пользователя и сообщения от сервера через select()
    \item Отправляет предположения серверу в формате GUESS
    \item Обрабатывает ответы сервера: результаты, состояние игры, победу
    \item Корректно завершает работу при отключении
\end{enumerate}

\subsection{Архитектура программы}

\begin{itemize}
    \item Сервер (server.c)
    \begin{itemize}
        \item Управляет игровой сессией
        \item Обрабатывает подключения и отключения игроков
        \item Вычисляет результаты предположений
        \item Использует select() для мультиплексирования ввода
    \end{itemize}
    \item Клиент (client.c)
    \begin{itemize}
        \item Подключается к серверу через именованные pipe'ы
        \item Обрабатывает пользовательский ввод и сообщения сервера
        \item Использует select() для одновременной работы с stdin и pipe
    \end{itemize}
    \item Логика игры (game\_logic.c)
    \begin{itemize}
        \item Генерация случайного слова из словаря
        \item Вычисление быков и коров
        \item Валидация предположений
    \end{itemize}
    \item Общие структуры (shared.h)
    \begin{itemize}
        \item Определения структур Player и Game
        \item Константы и макросы
        \item Макросы для логирования
    \end{itemize}
\end{itemize}

\subsection{Протокол обмена сообщениями}

Сообщения передаются через именованные pipe'ы в текстовом формате с разделителем \texttt{|}:

От клиента к серверу:
\begin{itemize}
    \item \texttt{CONNECT|player\_id|player\_name} - подключение к игре
    \item \texttt{GUESS|player\_id|слово} - предположение игрока
    \item \texttt{DISCONNECT|player\_id} - отключение от игры
\end{itemize}

От сервера к клиенту:
\begin{itemize}
    \item \texttt{CONNECT|player\_id|сообщение} - подтверждение подключения
    \item \texttt{RESULT|player\_id|быки|коровы|сообщение} - результат предположения
    \item \texttt{WIN|player\_id|слово|сообщение} - объявление победителя
    \item \texttt{GAME\_STATE|сообщение} - состояние игры
    \item \texttt{DISCONNECT|player\_id|сообщение} - уведомление об отключении игрока
\end{itemize}
