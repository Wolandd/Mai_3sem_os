\section{Выводы}

В ходе выполнения лабораторной работы были изучены и реализованы механизмы межпроцессного взаимодействия в операционной системе Linux:

\begin{enumerate}
    \item \textbf{Создание процессов}: Освоен системный вызов \texttt{fork()} для создания дочерних процессов и \texttt{execl()} для запуска отдельных программ.
    
    \item \textbf{Каналы (pipes)}: Реализована передача данных между процессами через каналы с использованием системных вызовов \texttt{pipe()} и \texttt{dup2()}.
    
    \item \textbf{Перенаправление потоков}: Применен системный вызов \texttt{dup2()} для перенаправления стандартных потоков ввода-вывода на каналы.
    
    \item \textbf{Обработка ошибок}: Реализована корректная обработка деления на ноль с уведомлением родительского процесса и завершением обоих процессов.
    
    \item \textbf{Неблокирующий режим}: Использован \texttt{fcntl()} для установки неблокирующего режима работы с каналами.
\end{enumerate}

Программа успешно выполняет все поставленные задачи и демонстрирует корректную работу механизмов межпроцессного взаимодействия через каналы. Реализованная система позволяет эффективно организовать взаимодействие между процессами с использованием стандартных механизмов операционной системы Linux.



