\section{Описание программы}

Программа состоит из двух файлов: \texttt{parent.c} (родительский процесс) и \texttt{child.c} (дочерний процесс).

Родительский процесс создает каналы через \texttt{pipe()}, создает дочерний процесс через \texttt{fork()}, перенаправляет потоки через \texttt{dup2()}, запускает программу \texttt{child} через \texttt{execl()}, читает команды от пользователя и передает их дочернему процессу.

Дочерний процесс получает имя файла из аргументов, открывает файл для записи, читает строки из stdin, парсит числа через \texttt{strtof()}, проверяет деление на ноль и записывает результат в файл.

\subsection{Используемые системные вызовы}

\begin{description}
    \item[\texttt{pipe(int pipefd[2])}] Создает канал и возвращает два файловых дескриптора: \texttt{pipefd[0]} для чтения и \texttt{pipefd[1]} для записи.
    
    \item[\texttt{fork()}] Создает копию текущего процесса. Возвращает PID дочернего процесса в родительском процессе и 0 в дочернем процессе.
    
    \item[\texttt{execl(const char *path, const char *arg, ...)}] Заменяет образ текущего процесса новой программой. Принимает путь к программе и аргументы командной строки.
    
    \item[\texttt{dup2(int oldfd, int newfd)}] Дублирует файловый дескриптор, позволяя перенаправить стандартные потоки (\texttt{stdin}, \texttt{stdout}) на каналы.
    
    \item[\texttt{write(int fd, const void *buf, size\_t count)}] Записывает данные в файловый дескриптор (в данном случае -- в канал).
    
    \item[\texttt{read(int fd, void *buf, size\_t count)}] Читает данные из файлового дескриптора (в данном случае -- из канала).
    
    \item[\texttt{waitpid(pid\_t pid, int *status, int options)}] Ожидает завершения дочернего процесса с указанным PID.
    
    \item[\texttt{fcntl(int fd, int cmd, ...)}] Выполняет различные операции с файловым дескриптором. Используется для установки неблокирующего режима через \texttt{F\_SETFL} и \texttt{O\_NONBLOCK}.
\end{description}

\subsection{Обработка ошибок}

Программа обрабатывает следующие системные ошибки:
\begin{itemize}
    \item Ошибки создания каналов (\texttt{pipe()})
    \item Ошибки создания процесса (\texttt{fork()})
    \item Ошибки перенаправления потоков (\texttt{dup2()})
    \item Ошибки запуска программы (\texttt{execl()})
    \item Ошибки записи в канал (\texttt{write()})
    \item Ошибки открытия файла (\texttt{fopen()})
    \item Деление на ноль (проверяется в дочернем процессе)
\end{itemize}

Все ошибки обрабатываются через проверку возвращаемых значений системных вызовов и вывод сообщений об ошибках через \texttt{perror()}.

