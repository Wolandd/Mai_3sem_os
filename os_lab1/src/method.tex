\section{Метод решения}

\subsection{Общее описание алгоритма}

Программа состоит из двух отдельных исполняемых файлов: \texttt{parent} (родительский процесс) и \texttt{child} (дочерний процесс).

Родительский процесс выполняет следующие действия:
\begin{enumerate}
    \item Запрашивает у пользователя имя файла для вывода результатов
    \item Создает два канала (pipe) для двусторонней связи с дочерним процессом
    \item Создает дочерний процесс через системный вызов \texttt{fork()}
    \item Перенаправляет стандартные потоки дочернего процесса на каналы через \texttt{dup2()}
    \item Запускает программу \texttt{child} через \texttt{execl()}
    \item В цикле читает команды от пользователя и передает их дочернему процессу через pipe
    \item Ожидает завершения дочернего процесса через \texttt{waitpid()}
\end{enumerate}

Дочерний процесс выполняет следующие действия:
\begin{enumerate}
    \item Получает имя файла из аргументов командной строки
    \item Открывает файл для записи результатов
    \item В цикле читает строки с числами из stdin (который перенаправлен на pipe)
    \item Парсит строку, извлекая числа типа float
    \item Проверяет наличие нуля среди делителей
    \item Выполняет деление первого числа на все последующие
    \item Записывает результат в файл или сообщает об ошибке при делении на ноль
\end{enumerate}

\subsection{Архитектура программы}

\begin{itemize}
    \item Родительский процесс (parent.c)
    \begin{itemize}
        \item Создает дочерний процесс через fork()
        \item Запускает child через execl()
    \end{itemize}
    \item Дочерний процесс (child.c)
    \begin{itemize}
        \item Читает данные из stdin (pipe\_to\_child)
        \item Пишет результаты в файл
        \item Отправляет сообщения в stdout (pipe\_from\_child)
    \end{itemize}
    \item Каналы:
    \begin{itemize}
        \item pipe\_to\_child: родитель -> дочерний (stdin)
        \item pipe\_from\_child: дочерний -> родитель (stdout)
    \end{itemize}
\end{itemize}

