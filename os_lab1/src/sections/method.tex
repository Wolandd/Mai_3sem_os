\section{Метод решения}

\subsection{Общее описание алгоритма}

Программа состоит из двух отдельных исполняемых файлов: \texttt{parent} (родительский процесс) и \texttt{child} (дочерний процесс).

Родительский процесс выполняет следующие действия:
\begin{enumerate}
    \item Запрашивает у пользователя имя файла для вывода результатов
    \item Создает два канала (pipe) для двусторонней связи с дочерним процессом
    \item Создает дочерний процесс через системный вызов \texttt{fork()}
    \item Перенаправляет стандартные потоки дочернего процесса на каналы через \texttt{dup2()}
    \item Запускает программу \texttt{child} через \texttt{execl()}
    \item В цикле читает команды от пользователя и передает их дочернему процессу через pipe
    \item Ожидает завершения дочернего процесса через \texttt{waitpid()}
\end{enumerate}

Дочерний процесс выполняет следующие действия:
\begin{enumerate}
    \item Получает имя файла из аргументов командной строки
    \item Открывает файл для записи результатов
    \item В цикле читает строки с числами из stdin (который перенаправлен на pipe)
    \item Парсит строку, извлекая числа типа float
    \item Проверяет наличие нуля среди делителей
    \item Выполняет деление первого числа на все последующие
    \item Записывает результат в файл или сообщает об ошибке при делении на ноль
\end{enumerate}

\subsection{Архитектура программы}

Схема взаимодействия процессов представлена на рисунке~\ref{fig:architecture}.

\begin{figure}[H]
\centering
\begin{verbatim}
┌─────────────────┐
│  Родительский   │
│    процесс      │
│   (parent.c)    │
└────────┬────────┘
         │
         │ fork()
         ▼
┌─────────────────┐
│  Дочерний       │
│    процесс      │
│   (child.c)     │
└────────┬────────┘
         │
         │ execl()
         ▼
┌─────────────────┐
│  Программа      │
│     child       │
└─────────────────┘

pipe_to_child[1] ──────> pipe_to_child[0] (stdin)
(родитель пишет)         (дочерний читает)

pipe_from_child[1] ──────> pipe_from_child[0]
(stdout дочернего)        (родитель читает)
\end{verbatim}
\caption{Архитектура взаимодействия процессов}
\label{fig:architecture}
\end{figure}

Основные системные вызовы, используемые в программе:
\begin{itemize}
    \item \texttt{pipe()} --- создание канала для межпроцессного взаимодействия
    \item \texttt{fork()} --- создание дочернего процесса
    \item \texttt{execl()} --- замена образа процесса новой программой
    \item \texttt{dup2()} --- перенаправление стандартных потоков на каналы
    \item \texttt{write()}, \texttt{read()} --- запись и чтение данных через каналы
    \item \texttt{waitpid()} --- ожидание завершения дочернего процесса
    \item \texttt{fcntl()} --- установка неблокирующего режима для канала
\end{itemize}



