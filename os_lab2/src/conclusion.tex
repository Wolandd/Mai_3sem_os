\section{Выводы}

В работе реализован многопоточный поиск минимального и максимального элементов массива средствами POSIX threads. Получено:
\begin{enumerate}
    \item Освоены базовые приёмы создания потоков (\texttt{pthread\_create}) и ожидания их завершения (\texttt{pthread\_join}).
    \item Показано, что разделение массива на независимые блоки позволяет обойтись без синхронизации при вычислении локальных результатов.
    \item Экспериментально подтверждено ускорение до числа аппаратных ядер; при избыточном числе потоков выигрыш исчезает из-за накладных расходов.
    \item Продемонстрирована важность выбора числа потоков относительно размера задачи: для маленьких входов выгоден один поток.
\end{enumerate}

Программа соответствует поставленному варианту: находит min/max в большом массиве, ограничивает число одновременно работающих потоков и позволяет оценить ускорение.
