\section{Результаты}

Сборка: \texttt{gcc -pthread main.c find\_min\_max.c -o find\_min\_max}

Пример запуска: \texttt{./find\_min\_max 1000000 4}

\subsection{Тесты}
\begin{itemize}
    \item \textbf{Малый массив (10 элементов, 2 потока):} результаты min/max совпадают с однопоточным вычислением.
    \item \textbf{1\,000\,000 элементов, 4 потока:} корректные min/max, время около 0.01--0.03 с (зависит от среды).
    \item \textbf{Количество потоков = 1:} совпадает по результату, служит базой для оценки ускорения.
\end{itemize}

\subsection{Наблюдения по ускорению}
\begin{itemize}
    \item При увеличении потоков до числа аппаратных ядер наблюдается ускорение за счёт распараллеливания.
    \item Дальнейшее увеличение потоков не даёт выигрыша из-за накладных расходов на создание и планирование.
    \item Для очень маленьких массивов выгоднее 1 поток из-за фиксированных накладных расходов.
\end{itemize}