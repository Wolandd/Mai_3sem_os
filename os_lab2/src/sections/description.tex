\section{Описание программы}

\subsection{Разделение по файлам}

Программа состоит из двух файлов:
\begin{itemize}
    \item \texttt{parent.c} --- родительский процесс, управляющий взаимодействием
    \item \texttt{child.c} --- дочерний процесс, выполняющий вычисления
\end{itemize}

\subsection{Основные типы данных}

В программе используются следующие типы данных:
\begin{itemize}
    \item \texttt{char filename[256]} --- буфер для имени файла
    \item \texttt{int pipe\_to\_child[2]}, \texttt{int pipe\_from\_child[2]} --- массивы дескрипторов каналов
    \item \texttt{pid\_t pid} --- идентификатор процесса
    \item \texttt{char line[512]} --- буфер для вводимой строки
    \item \texttt{float numbers[128]} --- массив для хранения чисел
\end{itemize}

\subsection{Основные функции}

\subsubsection{Родительский процесс (parent.c)}

\begin{itemize}
    \item \texttt{main()} --- основная функция, реализующая логику родительского процесса:
    \begin{itemize}
        \item Создание каналов через \texttt{pipe()}
        \item Создание дочернего процесса через \texttt{fork()}
        \item Перенаправление потоков через \texttt{dup2()}
        \item Запуск дочерней программы через \texttt{execl()}
        \item Цикл чтения команд и передачи их дочернему процессу
        \item Ожидание завершения дочернего процесса
    \end{itemize}
\end{itemize}

\subsubsection{Дочерний процесс (child.c)}

\begin{itemize}
    \item \texttt{main(int argc, char *argv[])} --- основная функция дочернего процесса:
    \begin{itemize}
        \item Получение имени файла из аргументов
        \item Открытие файла для записи
        \item Цикл чтения строк из stdin
        \item Парсинг строки и извлечение чисел через \texttt{strtof()}
        \item Проверка деления на ноль
        \item Выполнение деления и запись результата в файл
    \end{itemize}
\end{itemize}

\subsection{Используемые системные вызовы}

\begin{description}
    \item[\texttt{pipe(int pipefd[2])}] Создает канал и возвращает два файловых дескриптора: \texttt{pipefd[0]} для чтения и \texttt{pipefd[1]} для записи.
    
    \item[\texttt{fork()}] Создает копию текущего процесса. Возвращает PID дочернего процесса в родительском процессе и 0 в дочернем процессе.
    
    \item[\texttt{execl(const char *path, const char *arg, ...)}] Заменяет образ текущего процесса новой программой. Принимает путь к программе и аргументы командной строки.
    
    \item[\texttt{dup2(int oldfd, int newfd)}] Дублирует файловый дескриптор, позволяя перенаправить стандартные потоки (stdin, stdout) на каналы.
    
    \item[\texttt{write(int fd, const void *buf, size_t count)}] Записывает данные в файловый дескриптор (в данном случае --- в канал).
    
    \item[\texttt{read(int fd, void *buf, size_t count)}] Читает данные из файлового дескриптора (в данном случае --- из канала).
    
    \item[\texttt{waitpid(pid\_t pid, int *status, int options)}] Ожидает завершения дочернего процесса с указанным PID.
    
    \item[\texttt{fcntl(int fd, int cmd, ...)}] Выполняет различные операции с файловым дескриптором. Используется для установки неблокирующего режима через \texttt{F\_SETFL} и \texttt{O\_NONBLOCK}.
\end{description}

\subsection{Обработка ошибок}

Программа обрабатывает следующие системные ошибки:
\begin{itemize}
    \item Ошибки создания каналов (\texttt{pipe()})
    \item Ошибки создания процесса (\texttt{fork()})
    \item Ошибки перенаправления потоков (\texttt{dup2()})
    \item Ошибки запуска программы (\texttt{execl()})
    \item Ошибки записи в канал (\texttt{write()})
    \item Ошибки открытия файла (\texttt{fopen()})
    \item Деление на ноль (проверяется в дочернем процессе)
\end{itemize}

Все ошибки обрабатываются через проверку возвращаемых значений системных вызовов и вывод сообщений об ошибках через \texttt{perror()}.


