\section{Результаты}

\subsection{Компиляция программы}

Программа компилируется следующими командами:

\begin{lstlisting}[language=bash]
gcc parent.c -o parent
gcc child.c -o child
\end{lstlisting}

\subsection{Тестирование программы}

\subsubsection{Тест 1: Нормальная работа}

\textbf{Ввод:}
\begin{verbatim}
Введите имя файла для вывода результатов: result.txt
Введите строки вида: "число число число" (float). Пустая строка или EOF — завершение.
> 100 2 5
> 50 2.5
> 
\end{verbatim}

\textbf{Содержимое файла result.txt:}
\begin{verbatim}
Полученные числа: 100 2 5
Результат: 10.000000

Полученные числа: 50 2.5
Результат: 20.000000
\end{verbatim}

\textbf{Проверка:} $100 / 2 / 5 = 10$ \checkmark, $50 / 2.5 = 20$ \checkmark

\subsubsection{Тест 2: Деление на ноль}

\textbf{Ввод:}
\begin{verbatim}
Введите имя файла для вывода результатов: result.txt
Введите строки вида: "число число число" (float). Пустая строка или EOF — завершение.
> 100 2 5
> 50 0 2
\end{verbatim}

\textbf{Содержимое файла result.txt:}
\begin{verbatim}
Полученные числа: 100 2 5
Результат: 10.000000

Числа на входе: 50 0 2
Ошибка: деление на ноль
\end{verbatim}

\textbf{Вывод программы:}
\begin{verbatim}
Деление на 0. Завершение.
Родительский процесс завершён.
\end{verbatim}

\textbf{Анализ:} Программа корректно обнаружила деление на ноль и завершила работу обоих процессов.

\subsubsection{Тест 3: Произвольное количество чисел}

\textbf{Ввод:}
\begin{verbatim}
> 1000 2 5 2 2
\end{verbatim}

\textbf{Содержимое файла result.txt:}
\begin{verbatim}
Полученные числа: 1000 2 5 2 2
Результат: 25.000000
\end{verbatim}

\textbf{Проверка:} $1000 / 2 / 5 / 2 / 2 = 25$ \checkmark

\subsection{Ключевые особенности решения}

\begin{enumerate}
    \item \textbf{Разделение процессов}: Родительский и дочерний процессы представлены отдельными программами, что обеспечивает модульность и возможность независимой компиляции.
    
    \item \textbf{Двусторонняя связь}: Использование двух каналов позволяет организовать двустороннее взаимодействие между процессами --- родитель передает команды, дочерний может отправлять сообщения об ошибках.
    
    \item \textbf{Перенаправление потоков}: Использование \texttt{dup2()} позволяет дочернему процессу работать со стандартными потоками ввода-вывода, не зная о существовании каналов.
    
    \item \textbf{Обработка деления на ноль}: Проверка выполняется в дочернем процессе перед выполнением деления, что предотвращает ошибки и обеспечивает корректное завершение обоих процессов.
    
    \item \textbf{Обработка ошибок}: Все системные вызовы проверяются на ошибки, что обеспечивает надежность программы.
\end{enumerate}
