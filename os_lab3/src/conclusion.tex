\section{Выводы}

В ходе выполнения лабораторной работы были изучены и реализованы механизмы межпроцессного взаимодействия в операционной системе Linux с использованием разделяемой памяти и сигналов:

\begin{enumerate}
    \item \textbf{Разделяемая память POSIX}: Освоены системные вызовы \texttt{shm\_open()}, \texttt{mmap()}, \texttt{munmap()} и \texttt{shm\_unlink()} для создания и управления объектами разделяемой памяти. Разделяемая память обеспечивает эффективную передачу данных между процессами без использования файлов или каналов.
    
    \item \textbf{Создание процессов}: Освоен системный вызов \texttt{fork()} для создания дочерних процессов и \texttt{execvp()} для запуска отдельных программ.
    
    \item \textbf{Синхронизация через сигналы}: Реализована синхронизация процессов с использованием сигналов SIGUSR1 и SIGUSR2. Сигналы позволяют координировать выполнение процессов без блокирующих операций.
    
    \item \textbf{Обработка сигналов}: Реализованы обработчики сигналов для координации работы процессов. Обработчик SIGCHLD позволяет отслеживать завершение дочернего процесса и корректно обрабатывать случаи ошибок.
    
    \item \textbf{Обработка ошибок}: Реализована корректная обработка всех системных ошибок с проверкой возвращаемых значений и выводом сообщений через \texttt{perror()}. Особое внимание уделено обработке деления на ноль и некорректного формата входных данных.
    
    \item \textbf{Управление ресурсами}: Реализована корректная очистка ресурсов (закрытие разделяемой памяти, удаление объектов) как при нормальном завершении, так и при ошибках, что предотвращает утечки ресурсов.
\end{enumerate}

Программа успешно выполняет все поставленные задачи и демонстрирует корректную работу механизмов межпроцессного взаимодействия через разделяемую память и сигналы. Реализованная система позволяет эффективно организовать взаимодействие между процессами с использованием стандартных механизмов операционной системы Linux.

Основные преимущества использованного подхода:
\begin{itemize}
    \item Эффективность: разделяемая память обеспечивает быструю передачу данных без копирования
    \item Гибкость: сигналы позволяют реализовать сложные схемы синхронизации
    \item Надёжность: корректная обработка ошибок и очистка ресурсов
    \item Модульность: разделение на отдельные программы упрощает разработку и отладку
\end{itemize}
