\section{Условие}

\subsection{Цель работы}

Приобретение практических навыков в:
\begin{enumerate}
    \item Управлении процессами в ОС
    \item Обеспечении обмена данных между процессами посредством разделяемой памяти (memory-mapped files)
    \item Использовании сигналов для синхронизации процессов
\end{enumerate}

\subsection{Задание}

Составить программу на языке C для ОС Linux (запуск через WSL), которая реализует взаимодействие между родительским и дочерним процессами с использованием разделяемой памяти (memory-mapped files) и сигналов.

Программа должна состоять из следующих файлов:
\begin{itemize}
    \item \texttt{common.h} --- общие константы, структуры данных, прототипы функций
    \item \texttt{parent.c} --- родительский процесс
    \item \texttt{child.c} --- дочерний процесс
    \item \texttt{Makefile} --- сборка проекта
\end{itemize}
