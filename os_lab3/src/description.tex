\section{Описание программы}

\subsection{Разделение по файлам}

Программа состоит из трёх основных файлов:
\begin{itemize}
    \item \texttt{common.h} --- общие константы, структуры данных и прототипы функций
    \item \texttt{parent.c} --- родительский процесс, управляющий взаимодействием
    \item \texttt{child.c} --- дочерний процесс, выполняющий вычисления
\end{itemize}

\subsection{Основные константы и структуры}

В файле \texttt{common.h} определены:
\begin{itemize}
    \item \texttt{SHM\_NAME} --- имя объекта разделяемой памяти: \texttt{"/shm"}
    \item \texttt{SHM\_SIZE} --- размер буфера разделяемой памяти: 1024 байта
    \item \texttt{MAX\_FILENAME\_LEN} --- максимальная длина имени файла: 256 символов
    \item Константы сигналов: \texttt{SIGNAL\_READY} (SIGUSR1), \texttt{SIGNAL\_START} (SIGUSR2), \texttt{SIGNAL\_CHILD\_EXIT} (SIGCHLD)
    \item Прототипы функций для работы с разделяемой памятью
\end{itemize}

\subsection{Основные функции}

\subsubsection{Родительский процесс (parent.c)}

\begin{itemize}
    \item \texttt{create\_shared\_memory(const char* name, size\_t size)} --- создаёт объект разделяемой памяти, устанавливает размер и отображает его в адресное пространство процесса
    \item \texttt{close\_shared\_memory(void* addr, size\_t size)} --- закрывает отображение разделяемой памяти
    \item \texttt{unlink\_shared\_memory(const char* name)} --- удаляет объект разделяемой памяти
    \item \texttt{handle\_child\_done(int sig)} --- обработчик сигнала SIGUSR1, различает сигнал о готовности и сигнал о завершении
    \item \texttt{handle\_child\_exit(int sig)} --- обработчик сигнала SIGCHLD, отслеживает завершение дочернего процесса и сохраняет код завершения
    \item \texttt{main()} --- основная функция:
    \begin{itemize}
        \item Запрашивает имя файла и числа у пользователя
        \item Создаёт разделяемую память
        \item Устанавливает обработчики сигналов
        \item Создаёт дочерний процесс через \texttt{fork()} и запускает \texttt{execvp()}
        \item Записывает числа в разделяемую память
        \item Синхронизируется с дочерним процессом через сигналы
        \item Очищает ресурсы
    \end{itemize}
\end{itemize}

\subsubsection{Дочерний процесс (child.c)}

\begin{itemize}
    \item \texttt{open\_shared\_memory(const char* name, size\_t size)} --- открывает существующий объект разделяемой памяти и отображает его в адресное пространство процесса
    \item \texttt{close\_shared\_memory(void* addr, size\_t size)} --- закрывает отображение разделяемой памяти
    \item \texttt{handle\_start(int sig)} --- обработчик сигнала SIGUSR2, устанавливает флаг начала вычислений
    \item \texttt{main(int argc, char* argv[])} --- основная функция:
    \begin{itemize}
        \item Получает имя файла из аргументов
        \item Открывает разделяемую память
        \item Устанавливает обработчик SIGUSR2
        \item Отправляет сигнал о готовности родителю
        \item Ожидает сигнал SIGUSR2
        \item Читает и парсит числа из разделяемой памяти
        \item Выполняет последовательное деление с проверкой на ноль
        \item Записывает результат в файл
        \item Отправляет сигнал о завершении родителю
    \end{itemize}
\end{itemize}

\subsection{Используемые системные вызовы}

\begin{description}
    \item[\texttt{shm\_open(const char* name, int oflag, mode\_t mode)}] Создаёт или открывает объект разделяемой памяти POSIX. Возвращает файловый дескриптор.
    
    \item[\texttt{ftruncate(int fd, off\_t length)}] Устанавливает размер объекта разделяемой памяти.
    
    \item[\texttt{mmap(void* addr, size\_t length, int prot, int flags, int fd, off\_t offset)}] Отображает разделяемую память в адресное пространство процесса. Возвращает указатель на отображённую область.
    
    \item[\texttt{munmap(void* addr, size\_t length)}] Закрывает отображение разделяемой памяти.
    
    \item[\texttt{shm\_unlink(const char* name)}] Удаляет объект разделяемой памяти по имени.
    
    \item[\texttt{fork()}] Создаёт копию текущего процесса. Возвращает PID дочернего процесса в родительском процессе и 0 в дочернем процессе.
    
    \item[\texttt{execvp(const char* file, char* const argv[])}] Заменяет образ текущего процесса новой программой. Принимает имя программы и массив аргументов.
    
    \item[\texttt{signal(int signum, sighandler\_t handler)}] Устанавливает обработчик сигнала.
    
    \item[\texttt{kill(pid\_t pid, int sig)}] Отправляет сигнал процессу с указанным PID.
    
    \item[\texttt{pause()}] Приостанавливает выполнение процесса до получения сигнала.
    
    \item[\texttt{waitpid(pid\_t pid, int* status, int options)}] Ожидает завершения дочернего процесса с указанным PID. С опцией WNOHANG не блокирует выполнение.
    
    \item[\texttt{strtof(const char* nptr, char** endptr)}] Преобразует строку в число типа float.
\end{description}

\subsection{Обработка ошибок}

Программа обрабатывает следующие системные ошибки:
\begin{itemize}
    \item Ошибки создания/открытия разделяемой памяти (\texttt{shm\_open()})
    \item Ошибки установки размера разделяемой памяти (\texttt{ftruncate()})
    \item Ошибки отображения разделяемой памяти (\texttt{mmap()})
    \item Ошибки создания процесса (\texttt{fork()})
    \item Ошибки запуска программы (\texttt{execvp()})
    \item Ошибки отправки сигналов (\texttt{kill()})
    \item Ошибки открытия файла (\texttt{fopen()})
    \item Ошибки записи в файл (\texttt{fprintf()})
    \item Деление на ноль (проверяется в дочернем процессе перед выполнением операции)
    \item Некорректный формат чисел (проверяется при парсинге)
\end{itemize}

Все ошибки обрабатываются через проверку возвращаемых значений системных вызовов и вывод сообщений об ошибках через \texttt{perror()}. При ошибках выполняется корректная очистка ресурсов (закрытие разделяемой памяти, удаление объектов) перед завершением процесса.

\subsection{Механизм синхронизации}

Синхронизация между процессами реализована через сигналы и флаги:
\begin{enumerate}
    \item Дочерний процесс отправляет SIGUSR1 родителю о готовности
    \item Родительский процесс ожидает этот сигнал перед отправкой команды на вычисления
    \item Родительский процесс отправляет SIGUSR2 дочернему процессу для начала вычислений
    \item Дочерний процесс ожидает этот сигнал перед чтением данных из разделяемой памяти
    \item Дочерний процесс отправляет SIGUSR1 родителю о завершении вычислений
    \item Родительский процесс ожидает этот сигнал перед завершением
    \item Обработчик SIGCHLD отслеживает завершение дочернего процесса и позволяет родителю корректно обработать случаи, когда дочерний процесс завершается с ошибкой
\end{enumerate}
