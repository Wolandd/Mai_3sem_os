\section{Метод решения}

\subsection{Общее описание алгоритма}

Программа состоит из двух отдельных исполняемых файлов: \texttt{parent} (родительский процесс) и \texttt{child} (дочерний процесс).

Родительский процесс выполняет следующие действия:
\begin{enumerate}
    \item Запрашивает у пользователя имя файла для вывода результатов
    \item Запрашивает числа с плавающей точкой
    \item Создаёт объект разделяемой памяти через \texttt{shm\_open()}
    \item Устанавливает размер разделяемой памяти через \texttt{ftruncate()}
    \item Отображает разделяемую память в своё адресное пространство через \texttt{mmap()}
    \item Устанавливает обработчики сигналов: SIGUSR1 (готовность/завершение дочернего процесса) и SIGCHLD (отслеживание завершения)
    \item Создаёт дочерний процесс через \texttt{fork()} и запускает программу \texttt{child} через \texttt{execvp()}
    \item Записывает введённые числа в разделяемую память как строку
    \item Ожидает сигнал SIGUSR1 от дочернего процесса о готовности
    \item Отправляет сигнал SIGUSR2 дочернему процессу для начала вычислений
    \item Ожидает сигнал SIGUSR1 от дочернего процесса о завершении вычислений
    \item Очищает ресурсы: закрывает разделяемую память через \texttt{munmap()} и удаляет объект через \texttt{shm\_unlink()}
\end{enumerate}

Дочерний процесс выполняет следующие действия:
\begin{enumerate}
    \item Получает имя файла из аргументов командной строки
    \item Открывает существующий объект разделяемой памяти через \texttt{shm\_open()}
    \item Отображает разделяемую память в своё адресное пространство через \texttt{mmap()}
    \item Устанавливает обработчик сигнала SIGUSR2
    \item Отправляет сигнал SIGUSR1 родительскому процессу о готовности
    \item Ожидает сигнал SIGUSR2 от родительского процесса
    \item Читает строку чисел из разделяемой памяти
    \item Парсит строку, извлекая числа типа float через \texttt{strtof()}
    \item Проверяет наличие нуля среди делителей перед выполнением деления
    \item Выполняет последовательное деление первого числа на все последующие
    \item Записывает результат в файл с точностью 6 знаков после запятой
    \item Отправляет сигнал SIGUSR1 родительскому процессу о завершении
    \item Очищает ресурсы: закрывает разделяемую память через \texttt{munmap()}
\end{enumerate}

Основные системные вызовы, используемые в программе:
\begin{itemize}
    \item \texttt{shm\_open()} --- создание/открытие объекта разделяемой памяти
    \item \texttt{ftruncate()} --- установка размера разделяемой памяти
    \item \texttt{mmap()} --- отображение разделяемой памяти в адресное пространство процесса
    \item \texttt{munmap()} --- закрытие отображения разделяемой памяти
    \item \texttt{shm\_unlink()} --- удаление объекта разделяемой памяти
    \item \texttt{fork()} --- создание дочернего процесса
    \item \texttt{execvp()} --- замена образа процесса новой программой
    \item \texttt{signal()} --- установка обработчика сигнала
    \item \texttt{kill()} --- отправка сигнала процессу
    \item \texttt{pause()} --- ожидание сигнала
    \item \texttt{waitpid()} --- ожидание завершения дочернего процесса
\end{itemize}

\subsection{Синхронизация}

Синхронизация между процессами осуществляется через сигналы:
\begin{itemize}
    \item \textbf{SIGUSR1}: используется дочерним процессом для уведомления родителя о готовности и о завершении вычислений
    \item \textbf{SIGUSR2}: используется родительским процессом для команды дочернему процессу на начало вычислений
    \item \textbf{SIGCHLD}: автоматически отправляется системе при завершении дочернего процесса, используется родителем для отслеживания статуса завершения
\end{itemize}

Разделяемая память обеспечивает передачу данных между процессами без использования каналов или файлов. Оба процесса отображают один и тот же объект разделяемой памяти в своё адресное пространство, что позволяет им читать и записывать данные в общую область памяти.
