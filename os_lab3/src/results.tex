\section{Результаты}

\subsection{Компиляция программы}

Программа компилируется с помощью Makefile:

\begin{lstlisting}[language=bash]
make
\end{lstlisting}

Или вручную:
\begin{lstlisting}[language=bash]
gcc -Wall -Wextra -std=c11 -o parent parent.c -lrt
gcc -Wall -Wextra -std=c11 -o child child.c -lrt
\end{lstlisting}

Флаг \texttt{-lrt} необходим для линковки библиотеки POSIX shared memory.

\subsection{Тестирование программы}

\subsubsection{Тест 1: Нормальная работа}

\textbf{Ввод:}
\begin{verbatim}
$ ./parent
Enter result file name: output.txt
Enter numbers (separator - space): 12.5 2 0.5 5
Child process completed successfully.
Result written to file.
\end{verbatim}

\textbf{Содержимое файла output.txt:}
\begin{verbatim}
1.250000
\end{verbatim}

\textbf{Проверка:} $12.5 / 2 / 0.5 / 5 = 1.25$

\subsection{Ключевые особенности решения}

\begin{enumerate}
    \item \textbf{Разделяемая память}: Использование POSIX shared memory (\texttt{shm\_open}, \texttt{mmap}) обеспечивает эффективную передачу данных между процессами без использования файлов или каналов.
    
    \item \textbf{Синхронизация через сигналы}: Использование сигналов SIGUSR1 и SIGUSR2 для синхронизации процессов позволяет координировать выполнение без блокирующих операций.
    
    \item \textbf{Обработка завершения дочернего процесса}: Использование обработчика SIGCHLD позволяет родительскому процессу корректно обрабатывать случаи, когда дочерний процесс завершается с ошибкой (например, при делении на ноль).
    
    \item \textbf{Корректная очистка ресурсов}: Программа корректно закрывает разделяемую память через \texttt{munmap()} и удаляет объекты через \texttt{shm\_unlink()} как при нормальном завершении, так и при ошибках.
    
    \item \textbf{Обработка ошибок}: Все системные вызовы проверяются на ошибки, что обеспечивает надёжность программы и предотвращает утечки ресурсов.
    
    \item \textbf{Разделение процессов}: Родительский и дочерний процессы представлены отдельными программами, что обеспечивает модульность и возможность независимой компиляции.
\end{enumerate}

\subsection{Проверка работы разделяемой памяти}

Для проверки существования объекта разделяемой памяти можно использовать команду:
\begin{lstlisting}[language=bash]
ls -l /dev/shm/
\end{lstlisting}

После завершения программы объект \texttt{/shm} должен быть удалён благодаря вызову \texttt{shm\_unlink()} в родительском процессе.
