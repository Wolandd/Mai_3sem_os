\section{Выводы}

В ходе выполнения лабораторной работы были изучены и реализованы механизмы работы с динамическими библиотеками в операционной системе Linux:

\begin{enumerate}
    \item \textbf{Статическая линковка}: Освоена статическая линковка программы с динамической библиотекой на этапе компиляции. Программа \texttt{program1} демонстрирует прямой вызов функций из библиотеки \texttt{libimpl1.so}.
    
    \item \textbf{Динамическая загрузка}: Реализована динамическая загрузка библиотек во время выполнения с использованием системных вызовов \texttt{dlopen()}, \texttt{dlsym()} и \texttt{dlclose()}. Программа \texttt{program2} демонстрирует возможность переключения между различными реализациями функций.
    
    \item \textbf{Создание динамических библиотек}: Изучен процесс создания динамических библиотек (shared libraries) с использованием CMake и компилятора GCC. Библиотеки \texttt{libimpl1.so} и \texttt{libimpl2.so} содержат различные реализации одних и тех же функций.
    
    \item \textbf{Обработка ошибок}: Реализована корректная обработка ошибок при загрузке библиотек и получении символов через проверку возвращаемых значений и использование \texttt{dlerror()}.
    
    \item \textbf{Вычисление интегралов}: Реализованы два метода численного интегрирования: метод прямоугольников (impl1) и метод трапеций (impl2) для вычисления интеграла функции sin(x).
    
    \item \textbf{Системы счисления}: Реализованы функции перевода чисел в двоичную (impl1) и троичную (impl2) системы счисления с обработкой нуля и отрицательных чисел.
\end{enumerate}

Программа успешно выполняет все поставленные задачи и демонстрирует корректную работу механизмов статической и динамической линковки с библиотеками. Реализованная система позволяет эффективно организовать модульную архитектуру программы с возможностью переключения между различными реализациями функций во время выполнения.



