\section{Условие}

\subsection{Цель работы}

Целью является приобретение практических навыков в:
\begin{enumerate}
    \item Создание динамических библиотек
    \item Создание программ, которые используют функции динамических библиотек
\end{enumerate}

\subsection{Задание}

Требуется создать динамические библиотеки, которые реализуют заданный вариантом функционал. Далее использовать данные библиотеки 2-мя способами:
1.	Во время компиляции (на этапе «линковки»/linking)
2.	Во время исполнения программы. Библиотеки загружаются в память с помощью интерфейса ОС для работы с динамическими библиотеками
В конечном итоге, в лабораторной работе необходимо получить следующие части:
1.	Динамические библиотеки, реализующие контракты, которые заданы вариантом;
2.	Тестовая программа (программа №1), которая используют одну из библиотек, используя информацию полученные на этапе компиляции;
3.	Тестовая программа (программа №2), которая загружает библиотеки, используя только их относительные пути и контракты.
Провести анализ двух типов использования библиотек.

Пользовательский ввод для обоих программ должен быть организован следующим образом:
1.	Если пользователь вводит команду «0», то программа переключает одну реализацию контрактов на другую (необходимо только для программы №2). Можно реализовать лабораторную работу без данной функции, но максимальная оценка в этом случае будет «хорошо»;
2.	«1 arg1 arg2 … argN», где после «1» идут аргументы для первой функции, предусмотренной контрактами. После ввода команды происходит вызов первой функции, и на экране появляется результат её выполнения;
3.	«2 arg1 arg2 … argM», где после «2» идут аргументы для второй функции, предусмотренной контрактами. После ввода команды происходит вызов второй функции, и на экране появляется результат её выполнения.

\subsection{Вариант 14}

Первая функция:
1. Описание: 	Расчет интеграла функции sin(x) на отрезке [A, B]
2. Сигнатура: Float SinIntegral(float A, float B, float e)
3. Реализация 1	Подсчет интеграла методом прямоугольников.
4. Реализация 2	Подсчет интеграла методом трапеций.

Вторая функция:
1. Описание: 		Перевод числа x из десятичной системы счисления в другую
2. Сигнатура: Char* translation(long x)
3. Реализация 1	Другая система счисления двоичная
4. Реализация 2 	Другая система счисления троичная



