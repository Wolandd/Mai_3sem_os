\section{Описание программы}

Проект состоит из двух динамических библиотек и двух программ, демонстрирующих статическую и динамическую линковку.

Библиотека \texttt{impl1} содержит функции для вычисления интеграла методом прямоугольников и перевода чисел в двоичную систему. Библиотека \texttt{impl2} содержит функции для вычисления интеграла методом трапеций и перевода чисел в троичную систему.

Программа \texttt{program1} статически линкуется с \texttt{libimpl1.so} на этапе компиляции и использует функции напрямую. Программа \texttt{program2} динамически загружает библиотеки во время выполнения и может переключаться между реализациями.

\subsection{Используемые системные вызовы и функции}

\begin{description}
    \item[\texttt{dlopen(const char *filename, int flag)}] Открывает динамическую библиотеку и возвращает handle для работы с ней. Флаг \texttt{RTLD\_LAZY} означает ленивую загрузку символов.
    
    \item[\texttt{dlsym(void *handle, const char *symbol)}] Получает адрес символа (функции) из загруженной библиотеки по его имени.
    
    \item[\texttt{dlclose(void *handle)}] Закрывает динамическую библиотеку и освобождает связанные ресурсы.
    
    \item[\texttt{dlerror()}] Возвращает строку с описанием последней ошибки, возникшей при работе с динамическими библиотеками, или NULL, если ошибок не было.
    
    \item[\texttt{sinf(float x)}] Вычисляет синус числа типа float. Используется в функциях вычисления интеграла.
    
    \item[\texttt{malloc(size\_t size)}] Выделяет память для строки результата в функции \texttt{translation()}.
    
    \item[\texttt{free(void *ptr)}] Освобождает память, выделенную через \texttt{malloc()}.
\end{description}

\subsection{Обработка ошибок}

Программа обрабатывает следующие ошибки:
\begin{itemize}
    \item Ошибки загрузки библиотеки (\texttt{dlopen()} возвращает NULL)
    \item Ошибки получения символов (\texttt{dlsym()} возвращает NULL)
    \item Ошибки выделения памяти (\texttt{malloc()} возвращает NULL)
    \item Некорректные параметры функций (A >= B, e <= 0)
    \item Некорректный пользовательский ввод
\end{itemize}

Все ошибки обрабатываются через проверку возвращаемых значений функций и вывод сообщений об ошибках через \texttt{fprintf(stderr, ...)} и \texttt{dlerror()}.

